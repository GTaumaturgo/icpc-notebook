\documentclass[a4paper,10pt,oneside]{article}
% \setcounter{secnumdepth}{-1} 

\usepackage{amsmath}
\usepackage{amssymb}
\usepackage{courier}
\usepackage{graphicx}
\usepackage{xcolor}
\usepackage{color}
\usepackage{pdflscape}
\usepackage[utf8]{inputenc}
\usepackage{listings}
\usepackage[inline]{enumitem}
\usepackage{verbatim}
\usepackage{pxfonts}

\usepackage[top=2cm, bottom=1.5cm, left=1cm, right=1cm]{geometry}
\usepackage{multicol}
\usepackage{fancyhdr}

\pagestyle{fancy}

\renewcommand{\sectionmark}[1]{\markboth{#1}{}}
\renewcommand{\subsectionmark}[1]{\markright{#1}}

\fancyhf{}
%\rhead{\fancyplain{}{\nouppercase{\rightmark}}, page \bfseries\thepage}
%\rhead{\leftmark\ \-- \rightmark}
\rhead{\leftmark}

%\lhead{É dificil xover um nome}
\lhead{University of Brasilia}
%\chead{É dificil xover um nome}
\cfoot{\thepage}

\usepackage{titlesec}
\titlespacing*{\section}
{0pt}{2ex}{1ex}

\definecolor{dkgray}{rgb}{0.4,0.4,0.4}
\definecolor{gray}{rgb}{0.6,0.6,0.6}

\lstset{
	language=c++,
	tabsize=4,
	%frame=tb,
	aboveskip=.1em,
	belowskip=.1em,
	showstringspaces=false,
	basicstyle={\small\ttfamily},
	columns=fullflexible,
	numbers=none,
	%keywordstyle=\bfseries,
	breaklines=true,
	breakindent=1.1em,
	breakatwhitespace=false,
	commentstyle=\color{gray},
}

\newcommand\includes[2]{
   \subsection{#1}
   \lstinputlisting{#2}
}

\setlength{\columnseprule}{1pt}

%\title{ACM ICPC Reference}
\title{100\% É Pouco,Pagode Importa D+}
\author{University of Brasilia}

\begin{document}
\maketitle
\tableofcontents
\newpage
\begin{multicols}{2}
\thispagestyle{fancy}

\lstinputlisting[language=bash]{vimrc}

\lstinputlisting[language=bash]{bashrc}

\section{Data Structures}
\includes{Merge Sort Tree}{code/ed/merge_sort_tree.cpp}
\includes{Wavelet Tree}{code/ed/wavelet_tree.cpp}
\includes{Ordered Set}{code/ed/ordered.cpp}
\includes{Convex Hull Trick}{code/ed/cht.cpp}
\includes{Min queue}{code/ed/minq.cpp}
\includes{Sparse Table}{code/ed/sparse_table.cpp}

\section{Paradigms}
\includes{FFT}{code/fft.cpp}
\includes{NTT}{code/ntt.cpp}

\section{Math}
\includes{Euclides Extendido}{code/math/euclides.cpp}
\includes{Preffix inverse}{code/math/inv.cpp}
\includes{Pollard Rho}{code/math/pollard_rho.cpp}
\includes{Miller Rabin}{code/math/miller_rabin.cpp}
\includes{Totiente}{code/math/tot.cpp}
\includes{Mulmod TOP}{code/math/mod.cpp}
\includes{Determinant}{code/math/det.cpp}

\section{Graphs}
\includes{Dinic}{code/graph/dinic.cpp}
\includes{Min Cost Max Flow}{code/graph/mcmf.cpp}
\includes{Small to Large}{code/graph/stl.cpp}
\includes{Junior e Falta de Ideias}{code/graph/centroid_decomp.cpp}
\includes{Kosaraju}{code/graph/kosaraju.cpp}
\includes{Tarjan}{code/graph/tarjan.cpp}
\includes{Max Clique}{code/graph/maxcliq.cpp}

\section{Strings}
\includes{Aho Corasick}{code/string/aho-corasick.cpp}
\includes{Suffix Array}{code/string/suffix_array.cpp}
\includes{Z Algorithm}{code/string/z_algo.cpp}
\includes{Prefix function/KMP}{code/string/pf.cpp}
\includes{Min rotation}{code/string/min_rot.cpp}
\includes{All palindrome}{code/string/all_palindrome.cpp}
\includes{Palindromic Tree}{code/string/ptree.cpp}

\section{Geometry}
\includes{2D basics}{code/geometry/2D.cpp}
\includes{Nearest Points}{code/geometry/near.cpp}
\includes{Convex Hull}{code/geometry/convexhull.cpp}
\includes{Check point inside polygon, borders included}{code/geometry/in_poly.cpp}

\subsection{Triangulo}
\textbf{Baricentro} (centro de massa), ponto de interseção entre as três medianas(segmentos de reta que unem um vértice ao ponto médio do lado oposto). Divide cada mediana na proporção de 2:1. As coordenadas são a média aritmética entre as coordenadas dos vértices.

O \textbf{ortocentro} de um triângulo é o ponto de encontro de suas três alturas(reta passando por um vértice e perpendicular ao lado oposto). O ortocentro pode mesmo estar fora do triângulo (no caso de um obtusângulo). No caso de um triângulo retângulo, o ortocentro sempre coincide com o vértice oposto à hipotenusa.

O simétrico do ortocentro em relação a qualquer um dos lados pertence ao circuncírculo.
O simétrico do ortocentro em relação a qualquer uma das medianas dos lados do triângulo também encontra-se sobre o circuncírculo.
Sendo H o ortocentro e O o circuncentro do triângulo ABC, o ângulo ABH = OAC.

O \textbf{incentro} de um triângulo é o ponto de encontro de suas bissetrizes (retas que dividem um ângulo interno na metade). Além de ser sempre um ponto interior do triângulo, o incentro é o centro do círculo inscrito no triângulo, isto é, o maior círculo que cabe dentro do triângulo e que toca todos os seus três lados (os lados são tangentes ao círculo inscrito).

O raio do círculo inscrito é dado pela razão entre o dobro da área e o perímetro. As coordenadas do centro O do círculo inscrito são obtidas pela média ponderada das coordenadas x e y pelos comprimentos dos lados opostos. As fórmulas abaixo sintetizam estas afirmações $p = a+b+c$.

$r = \frac{2A}{p}, Ox = \frac{a*A_x + b*B_x + c*C_x}{p}, Oy = \frac{a*A_y + b*B_y + c*C_y}{p}$

O \textbf{circuncentro} é o ponto de encontro entre as retas bisectoras perpendiculares (isto é, retas perpendiculares aos lados do triângulo que os interceptam nos pontos médios). O circuncentro é o centro do círculo circunscrito do triângulo, isto é, o círculo que passa pelos três vértices do triângulo.

O circuncentro, assim como o ortocentro, pode estar localizado do lado externo do triângulo. Um caso especial interessante é o do triângulo retângulo, onde o circuncentro se localiza no ponto médio da hipotenusa.

O raio do circuncentro é dado pela razão entre o produto das medidas de seus lados e o quádruplo de sua área. As coordenadas do circuncentro podem ser determinadas pelas expressões abaixo, onde $|A|^2 = Ax^2 + Ay^2$, ou $A*A$.

$r = \frac{a*b*c}{4A}, S_x = \frac{1}{2d} * \begin{vmatrix}
|A|^2 & A_y & 1 \\
|B|^2 & B_y & 1 \\
|C|^2 & C_y & 1 \\
\end{vmatrix}, S_y = \frac{1}{2d} * \begin{vmatrix}
A_x & |A|^2 & 1 \\
B_x & |B|^2 & 1 \\
C_x & |C|^2 & 1 \\
\end{vmatrix}$

$d = \begin{vmatrix}
A_x & A_y & 1 \\
B_x & B_y & 1 \\
C_x & C_y & 1 \\
\end{vmatrix}$

$\mid A\mid^2$ é dot product do vetor $A$ com si mesmo.

\section{Miscellaneous}
\includes{LIS}{code/misc/lis.cpp}
\includes{DSU rollback}{code/misc/bipar.cpp}

\end{multicols}
\end{document}
