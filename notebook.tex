\documentclass[a4paper,10pt,oneside]{article}
% \setcounter{secnumdepth}{-1} 

\usepackage{amsmath}
\usepackage{amssymb}
\usepackage{courier}
\usepackage{graphicx}
\usepackage{xcolor}
\usepackage{color}
\usepackage{pdflscape}
\usepackage[utf8]{inputenc}
\usepackage{listings}
\usepackage[inline]{enumitem}
\usepackage{verbatim}
\usepackage{pxfonts}
%\usepackage{algorithmic}
\usepackage{algorithm2e}
%\usepackage{algorithmicx}
%\usepackage{program}
%\usepackage{algorithm}
\usepackage[noend]{algpseudocode}

\usepackage[top=2cm, bottom=1.5cm, left=1cm, right=1cm]{geometry}
\usepackage{multicol}
\usepackage{fancyhdr}

\pagestyle{fancy}

\renewcommand{\sectionmark}[1]{\markboth{#1}{}}
\renewcommand{\subsectionmark}[1]{\markright{#1}}

\fancyhf{}
%\rhead{\fancyplain{}{\nouppercase{\rightmark}}, page \bfseries\thepage}
%\rhead{\leftmark\ \-- \rightmark}
\rhead{\leftmark, \thepage}

%\lhead{É dificil xover um nome}
\lhead{University of Brasilia}
%\chead{É dificil xover um nome}
\cfoot{\thepage}

\usepackage{titlesec}
\titlespacing*{\section}
{0pt}{2ex}{1ex}

%\definecolor{dkgray}{rgb}{0.4,0.4,0.4}
\definecolor{gray}{rgb}{0.6,0.6,0.6}
\definecolor{dkgreen}{rgb}{0,0.6,0}
%\definecolor{gray}{rgb}{0.5,0.5,0.5}
\definecolor{mauve}{rgb}{0.58,0,0.82}

\lstset{
	language=c++,
	tabsize=4,
	%frame=tb,
	aboveskip=.1em,
	belowskip=.1em,
	showstringspaces=false,
	basicstyle={\small\ttfamily},
	columns=fullflexible,
	numberstyle=\tiny\color{gray},
	keywordstyle=\color{blue},
	commentstyle=\color{dkgreen},
	stringstyle=\color{mauve},
	numbers=none,
	%keywordstyle=\bfseries,
	breaklines=true,
	breakindent=1.1em,
	breakatwhitespace=false,
	commentstyle=\color{gray},
}

\newcommand\includes[2]{
   \subsection{#1}
   \lstinputlisting{#2}
}

\setlength{\columnseprule}{1pt}

\date{}
%\title{100\% É Pouco,Pagode Importa D+}
\title{ICPC Team Reference}
\author{University of Brasilia}

\begin{document}
\maketitle
\begin{multicols}{2}
\tableofcontents
\newpage
\thispagestyle{fancy}

\lstinputlisting[language=bash]{vimrc}

\lstinputlisting[language=bash]{bashrc}

\section{Data Structures}
\includes{Merge Sort Tree}{code/ed/merge_sort_tree.cpp}
\includes{Wavelet Tree}{code/ed/wavelet_tree.cpp}
\includes{Order Set}{code/ed/order_set.cpp}
\includes{Hash table}{code/ed/hash_table.cpp}
\includes{Convex Hull Trick Simple}{code/ed/cht_simple.cpp}
\includes{Convex Hull Trick}{code/ed/cht.cpp}
\includes{Convex Hull Trick}{code/ed/LineContainer.cpp}
\includes{Min queue}{code/ed/minq.cpp}
\includes{Sparse Table}{code/ed/sparse_table.cpp}
\includes{Treap}{code/ed/treap.cpp}
\includes{ColorUpdate}{code/misc/ColorUpdate.cpp}
\includes{Heavy Light Decomposition}{code/ed/hld.cpp}
\includes{Iterative Segtree}{code/ed/segtree.cpp}
\includes{LiChao's Segtree}{code/ed/lichao.cpp}
\includes{Palindromic tree}{code/ed/eertree.cpp}


\section{Math}
\includes{Extended Euclidean Algorithm}{code/math/euclides.cpp}
\includes{Chinese Remainder Theorem}{code/math/crt.cpp}
\includes{Preffix inverse}{code/math/inv.cpp}
\includes{Pollard Rho}{code/math/pollard_rho.cpp}
\includes{Miller Rabin}{code/math/miller_rabin.cpp}
\includes{Totiente}{code/math/tot.cpp}
\includes{Mobius Function}{code/math/mobius.cpp}
\includes{Mulmod TOP}{code/math/mod.cpp}
\includes{Matrix Determinant}{code/math/det.cpp}
\includes{Simplex Method}{code/math/simplex.cpp}
\includes{FFT}{code/fft.cpp}
\includes{NTT}{code/ntt.cpp}

\section{Graphs}
\includes{Dinic}{code/graph/dinic.cpp}
\includes{Min Cost Max Flow}{code/graph/mcmf.cpp}
\includes{Small to Large}{code/graph/stl.cpp}
\includes{Junior e Falta de Ideias}{code/graph/centroid_decomp.cpp}
\includes{Kosaraju}{code/graph/kosaraju.cpp}
\includes{Tarjan}{code/graph/tarjan.cpp}
\includes{Max Clique}{code/graph/maxcliq.cpp}
\includes{Dominator Tree}{code/graph/dominator_tree.cpp}
\includes{Min Cost Matching}{code/graph/hungarian_mcm.cpp}

\section{Strings}
\includes{Aho Corasick}{code/string/aho_corasick.cpp}
\includes{Suffix Array}{code/string/suffix_array.cpp}
\includes{Z Algorithm}{code/string/z_algo.cpp}
\includes{Prefix function/KMP}{code/string/pf.cpp}
\includes{Min rotation}{code/string/min_rot.cpp}
\includes{All palindrome}{code/string/all_palindrome.cpp}
\includes{Manacher}{code/string/manacher.cpp}
\includes{Palindromic Tree}{code/string/ptree.cpp}
\includes{Suffix Automaton}{code/string/suffix_automaton.cpp}
\includes{Suffix Tree}{code/ed/suffix_tree.cpp}

\section{Geometry}
\includes{2D basics}{code/geometry/2D.cpp}
\includes{Nearest Points}{code/geometry/near.cpp}
\includes{Convex Hull}{code/geometry/convexhull.cpp}
\includes{Check point inside polygon}{code/geometry/in_poly.cpp}
\includes{Check point inside polygon without lower/upper hull}{code/geometry/in_poly2.cpp}
\includes{Minkowski sum}{code/geometry/mink.cpp}

\section{Miscellaneous}
\includes{LIS}{code/misc/lis.cpp}
\includes{DSU rollback}{code/misc/bipar.cpp}
\includes{Buildings}{code/misc/burn.cpp}
\includes{Rand}{code/misc/rand.cpp}
\includes{Klondike}{code/misc/klondike.cpp}
\includes{Hilbert Order}{code/misc/hilbert_order.cpp}
\includes{Modular Factorial}{code/misc/factmod.cpp}
\includes{Enumeration all submasks of a bitmask}{code/misc/submasks.cpp}
\includes{Slope Trick}{code/misc/slope.cpp}
\includes{Fast IO}{code/misc/fastio.cpp}
% \includes{Big int}{code/misc/bigint.cpp}
\includes{Knapsack Bounded with Cost}{code/misc/knapsack_bounded_cost.cpp}
\includes{LCA \textless O(nlgn), O(1)\textgreater}{code/misc/lca.cpp}


\subsection{Burnside's Lemma}
Let $(G, \oplus)$ be a finite group that acts on a set $X$. It should hold that $e_g*x=x$ and $g_1 *(g_2 * x) = (g_1 \oplus g_2) * x$, $\forall x \in X, g_1, g_2 \in G$. For each $g \in G$ let $X^g = \{x \in X \mid g*x = x \}$. The number of orbits its given by:

$\mid X / G\mid~= \frac{1}{|G|} \sum_{g \in G}{|X^g|}$

\subsection{Wilson's Theorem}
$(n-1)! = -1 \mod n \iff n\text{ is prime}$

\subsection{Fibonacci}
\begin{itemize}
\item $F_{n-1}F_{n+1} - F_n^2 = (-1)^n$
\item $F_{n+k} = F_kF_{n+1} + F_{k-1}F_n$
\item $GCD(F_n, F_m) = F_{GCD(n, m)}$
\item $F_n = \frac{(\frac{1+\sqrt{5}}{2})^n - (\frac{1-\sqrt{5}}{2})^n}{\sqrt{5}}$
\end{itemize}

\subsection{Kirchhoff's Theorem}
Laplacian matrix is $L = D - A$, where $D$ is a diagonal matrix with vertex degrees on the diagonals and $A$ is adjacency matrix.

The number of spanning trees is any cofactor of L. i-th cofactor is determinant of the matrix gotten by removing i-th row and column of L.

\subsubsection{Multigraphs}
In $D[i][i]$ all loops are excluded. $A[i][j]$ = number of edges from $i$ to $j$.

\subsubsection{Directed multigraphs}
$D[i][i]$ = indegree of i minus the number of loops at i. $A[i][j]$ = number of edges from $i$ to $j$.

The number of oriented spanning trees rooted at a vertex i is the determinant of the matrix gotten by removing the ith row and column of L.

\subsubsection{Matroid}
\subsection{Edge coloring}

\begin{algorithm}[H]
\SetAlgoLined
\KwData{A graph G}
\KwResult{A proper coloring c of the edges of G}
Let U := E(G)\;
\While{$U \not= \emptyset$}{
Let (u,v) be any edge in U\;
Let F[1:k] be a maximal fan of u starting at F[1]=v\;
Let c be a color that is free on u and d be a color that is free on F[k]\;
Invert the $cd_u$ path\;
Let $w \in V(G)$ be such that $w \in F, F'=[F[1]...w]$ is a fan and d is free on w\;
Rotate F' and set c(u,w)=d\;
U := U - {(u,v)}\;
}
\end{algorithm}

\subsection{Notes}
When we repeat something and each time we have probability $p$ to succeed then the expected number or tries is $\frac{1}{p}$, till we succeed.

\textbf{Small to large}

\textbf{Trick in statement} If $k$ sets are given you should note that the amount of different set sizes is $O(\sqrt{s})$ where $s$ is total size of those sets. And no more than $\sqrt{s}$ sets have size greater than $\sqrt{s}$. For example, a path to the root in Aho-Corasick through suffix links will have at most $O(\sqrt{s})$ vertices.

\textbf{gcd on subsegment}, we have at most $\log(a_i)$ different values in $\{\gcd(a_j, a_{j+1}, ..., a_i)$ for $j < i\}$.

\textbf{From static set to expandable}. To insert, create a new set with the new element. While there are two sets with same size, merge them. There will be at most $\log(n)$ disjoints sets.

\textbf{Matrix exponentiation optimization}. Save binary power of $A_{nxn}$ and answer $q$ queries $b = A^mx$ in $O((n^3 + qn^2)log(m))$.

\textbf{Ternary search on integers into binary search}, comparing f(mid) and f(mid+1), binary search on derivative

\textbf{Dynamic offline set} For each element we will wind segment of time $[a, b]$ such that element is present in the set during this whole segment. Now we can come up with recursive procedure which handles $[l, r]$ time segment considering that all elements such that $[l, r] \subset [a, b]$ are already included into the set. Now, keeping this invariant we recursively go into $[l, m]$ and $[m+1, r]$ subsegments. Finally when we come into segment of length 1.

$a > b \implies a \mod b < \frac{a}{2}$

\end{multicols}


\end{document}
